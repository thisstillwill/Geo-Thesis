\section{Introduction}

With roughly 4.5 billion users worldwide~\cite{umaine_2021}, the scale of social media is testament to the importance of digital communication today. At the same time, humans are social animals. It is here in the physical world that we spend most of our time and have our closest interactions with each other. While 99\% of users access social media through mobile devices~\cite{umaine_2021}, our interaction with these services is often still independent of the physical space around us.

Previous work~\cite{ducheneaut_2006} also suggests that digital interaction does not necessarily correspond to an improved social experience. This is especially interesting in light of the ongoing \textsc{COVID-19} pandemic and the move towards remote work. Research~\cite{bonsaksen_2021,mann_2003} points to a reliance on digital communication and social media as increasing social isolation.

There is much potential, however, in exploring the intersection between digital communication and physical space. Facebook, Twitter, and Instagram—among the largest social media platforms~\cite{umaine_2021}—now incorporate geospatial data into their products. Other platforms, like Yik Yak, even place location at the core of the user experience~\cite{yikyak_2015}.

The goal of this paper is to explore how geospatial data affects our perception of digital communication. My hypothesis was that tying interaction to physical proximity using would increase user engagement. To this end, I present a novel communication app that gates the sending and receiving of messages to the user's location.
