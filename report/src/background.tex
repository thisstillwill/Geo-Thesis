\section{Background}

\subsection{Geospatial Data}

Geospatial data is data that is related in some way to a specific geographic position. Such data typically combines location information in the form of coordinates, attribute information about the object or event in question, and temporal information connected to the time this data existed~\cite{ibm}. In this way, geospatial data connects digital information to a very real sense of time and place.

One of the principal uses of geospatial data is in geospatial analysis, utilizing geographic information systems (\textsc{GIS}) and related tools. Geospatial analysis can improve data visualization by providing additional context to traditional data analysis. Of particular relevance to this topic is the flexibility of geospatial data. While originally used in geology, epidemiology, and the life sciences, geospatial data can be applied to areas as diverse as defense and social science~\cite{ibm}.

\subsection{Proximity Principle}

In social psychology, the proximity principal relates the tendency to form social relationships with the physical distance between people. The phenomenon was first observed by \citeauthor{newcomb_1960}~\cite{newcomb_1960} and also explored by \citeauthor{festinger_1950}~\cite{festinger_1950}. \citeauthor{marmaros_2006}~\cite{marmaros_2006} surveyed students and recent graduates at Dartmouth College, and found that physical proximity was more influential than any other factor in determining the level of social interaction between people.
