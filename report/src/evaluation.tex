\section{Evaluation}

Given the exploratory nature of this project, I intended to evaluate the success of my app off the qualitative feedback from my users. My goal was to ask questions that would be able to confirm or reject my initial hypothesis.

\subsection{Logistics}

The evaluation period occurred over two days, starting on Wednesday, April 11th. I solicited participants from three Princeton email listservs (Princeton Tower Club, Computer Science undergraduates, and Princeton Silicon Valley TigerTrek). I included an explanation of the project, a demonstration video, and a public link to the iPhone app on TestFlight. Testers were asked to use the app over the following 48 hours and then fill out a feedback form hosted on Google Forms, after which they would be given a Starbucks gift card.

\subsection{Results}

19 unique users accepted the beta invite on TestFlight, of which 15 opened the app at least once. Participants used the app over an average of 8.4 sessions, reporting only 1 crash in total to TestFlight. The feedback form consisted of several questions gauging the general quality and purpose of the app. Out of a scale from 1 to 10, the average rating for overall user experience was roughly a 7.

Multiple users drew comparisons to Pokémon Go, an augmented reality game that also requires users to move in the real world in order to interact with the app. A majority agreed with the hypothesis that tying messages to physical locations made the user experience more engaging. A full 83.3\% of users felt the primary use case was sending status messages à la Twitter or microblogging services.

A common complaint centered around issues with \textsc{UX}. Despite my attempts to color-code which points were within range, many users expressed a preference for an explicit interaction radius that would be shown visually on the map screen. Additionally, users suggested that a clearer indicator was needed for when messages would expire. Finally, a universal issue was the lack of interaction with submitted messages. Many testers indicated it would be better to have features like a friends list and the ability to respond to submitted messages. This followed from a general sentiment that the low tester population made it difficult to feel like part of a broader community of users.

\subsection{Discussion}

Despite the negative points described above, I conclude that my project was ultimately successful. Some of the issues raised, especially regarding user population, are unfortunately a symptom of using a limited number of participants on short notice. An extended period of evaluation might be able to encourage a more organic community to form over time. Testing verified that the core functionality of my app worked as intended, and was at least intuitive enough to be learned quickly by new users. More importantly, participants enjoyed the geospatial aspects of the app. They felt motivated to interact with the environment and search out messages, either deliberately or passively during their normal day. This behavior confirms my initial hypothesis and the overall goals of this project.
