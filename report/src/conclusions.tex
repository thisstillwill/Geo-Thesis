\section{Conclusions and Future Work}

The application of geospatial data in digital communication remains a relatively unexplored area. It is still mostly a tool for contextual analysis or—as in social media—the organization of existing user content. This paper shows the potential of geospatial data as a means to connect users with both digital platforms and the world around them in new and interesting ways.

In spite of these conclusions, there is much more that can still be done. This project is ultimately a proof of concept for the user experience I originally envisioned. There is room to flesh out the application, both client and server, to match the feature quality of commercial products. In particular, there is much potential in improving the app's interactivity beyond leaving and viewing messages. A system to add comments to an existing message, for example, would enable users to continue the conversation and facilitate interaction between users and not just with the message itself.
