\section{Related Work}

\subsection{Geocaching}

Geocaching is one of the most popular activities that centers around the use of geospatial data. Participants search for hidden caches by using \textsc{GPS}, attempting to be the first to reach a given cache's location. The coordinates for the cache, as well as clues regarding its surroundings, are posted online by organizers beforehand~\cite{natgeo_2017} and can be freely discussed. A similar activity, called waymarking, exists that actually forgoes an actual physical object. The goal instead is to reach a \enquote{virtual cache} that represents an interesting trail, vista, or other location of interest~\cite{natgeo_2017}.

\subsection{Social Media}

As previously discussed, a number of existing social media platforms already use geospatial data. The extent and purpose of geospatial data varies heavily between different services.

\subsubsection{Instagram}

Instagram is a photo and video sharing platform that was acquired by Facebook, Inc. (now Meta) in 2012. Like Facebook, Instagram uses geospatial data in the form of geotagging, where geographic information is attached as metadata. Users who click or search for a location can view a feed of other content that was posted nearby. In other words, geospatial data is used to categorize and offer additional context to content.

\subsubsection{Twitter}

Similarly to Instagram, Twitter supports geotagging. However, this metadata is limited to images that users attach to Twitter messages, or \enquote{Tweets.} Twitter is also unique in that Tweets are limited in length and are by default text-only~\cite{twitter_2022}.

\subsubsection{Snapchat}

Snapchat is an instant messaging service focusing on \enquote{Snaps.} Snaps are photo, video, or text messages that by default last only 24 hours. Snapchat supports geotagging for both photo and video messages, and—unique to the platform—also allows messages to be sent to the app's \enquote{Snap Map} service. Snap Map, developed by Mapbox, shows a real-time heatmap of Snaps overlaid on the world map~\cite{tadesse_2017}. By tapping on part of the map, users are shown a feed of Snaps posted at that location.

\subsubsection{Yik Yak}

Of the platforms that have already been mentioned, Yik Yak is the only service that actually controls access to content using geospatial data. Yik Yak is a messaging app where individual posts, called \enquote{Yaks,} are submitted and viewed within a 5-mile radius~\cite{yikyak_2015}. This gates user interaction to the immediate community surrounding them.
